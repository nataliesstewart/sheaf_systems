\documentclass{amsart}  
\usepackage{../catstyle} 
\usepackage{../../../environments}
\begin{document}

\title{On double-negation sheaves in the copresheaf topos on a cofree comonoid}
\author{Natalie Stewart}
\maketitle

This note fixes a polynomial functor $p \in \Poly$ and studies the copresheaf topos $\sC_p := \sF_P$-$\Set$;
we fix $\Omega \in \sC_p$ the subobject classifier (given by the copresheaf of sieves on $\sF_p^{\op}$).

\section{Presheaves}
\subsection{$\sC_p$ as a topos of dynamical systems}
A presheaf $F \in \sF_p$ consists of the following data (subject to no restrictions):
\begin{enumerate}
  \item For each $p$-tree $i \in \sF_p$, a set $F(i)$.
  \item For each morphism $f:i \rightarrow j$ in $\sF_p$ corresponding with an inclusion of a subtree at height one, a function $F(f):F(i) \rightarrow F(j)$. 
\end{enumerate}
This corresponds with a ``dynamical system'' with ``positions with future knowledge'' corresponding with $p$-trees, states in each position corresponding with each set $F(i)$, and ``transition functions'' between the states in these positions where one moves to a possible ``next state'' according to the futures predicted by the $p$-tree structure.\footnote{\color{magenta} This sentence makes no sense.}
I will henceforth refer to a presheaf $F \in \sF_p$ as a \emph{$p$-dynamical system}.

\subsection{Subsystems}
Note that a morphism of $p$-dynamical systems $F \rightarrow G$ is precisely a map $F(i) \rightarrow G(i)$ from the $i$-states of $F$ to the $i$-states of $G$ for each $i \in \sF_p$, compatible with the transition functions in $F$ and $G$.
A morphism is monic iff each of the constituent morphisms are monic;
that is, a subobject corresponds with a subset of the states which is preserved under the transition functions.

As with any topos, the set of subobjects on a $p$-dynamical system $\Sub(F)$ comes equipped with a Heyting algebra stricture;
the join and meet are given by union and intersection, and the implication $A \implies B$ is given by \textbf{FILL IN THE RELEVANT CHARACTERIZATION}.
In particular, the \emph{negation} $\neg A = (A \implies 0)$ is given by the largest dynamical system contained in the set-theoretic complement of $A$, and hence $\neg \neq A$ is given by the largest dynamical system whose complement is the same as $A$.

As with any topos, $\neg \neg:\Omega \rightarrow \Omega$ is a Lawvere-Tierney topology on $\sC_p$.
We will seek to characterize the sheaves with respect to this topology, for which it will be useful to name the following dynamical (sub)systems.

\begin{example}
  Let $s \in F(i)$ be a state in a dynamical system.
  Define the system
  \[
    U_{s}(j) = \begin{cases}
      1 & \;\;\; \exists f:i \rightarrow j,\\
      0 & \;\;\; \text{otherwise.} 
    \end{cases}
  \]
  with transition functions are canonically defined.
  There is a unique monic $U_s \rightarrowtail F$ sending $U_{s}(i)$ to $s$.
  \begin{lemma}
    Any dynamical system $F$ is generated by the associated subsystems $U_{s}$;
    that is, 
    \[
      F = \bigcup_{s \in \coprod_i F(i)} U_{s}.
    \]
  \end{lemma}
\end{example}

It is perhaps troublesome that $U_s$ is defined in reference to a particular state of a dynamical system, as the systems themselves depend only on the position that the state resides in.
We can simply define $U_i$ for position $i$ by the above description;
these are contained in the following example.
\begin{example}
  The \emph{all-ones system} $1_p$ has $1_p(i) = \cbr{i}$ for all $i \in \sF_p$, with transition functions given by the unique endomorphism of 1.
  Note that there are unique monics $U_i \rightarrowtail V_i \rightarrowtail 1_p$.
  
  {\color{magenta} Consider mentioning that they generate all subobjects under union--they always do, but this time it's particularly simple.}
\end{example}


\section{Sheaves}\label{Sheaves section}
\subsection{Density}
The following lemma is well known:
\begin{lemma}
  In a presheaf topos, a subobject $A \subset C$ is $\neg\neg$-dense iff all nonzero subobjects $0 \subsetneq B \subset C$ intersect $A$.
\end{lemma}
This powers the following (easy) proposition, which shows that dense subsystems correspond intuitively with \emph{attainable win conditions}:
\begin{proposition}\label{Dense proposition}
  A subsystem $F \subset G$ is dense iff there is no state $s \in F(i)$ and sequence of morphisms $i = i_0 \xrightarrow{f_0} i_1 \xrightarrow{f_1} i_2 \xrightarrow{f_2} \cdots$ such that $f_n(s) \not \in G$ for all $n \in \NN$. 
\end{proposition}
\begin{proof}
  {\color{magenta} THIS NEEDS TO BE FIXED, AND IS ACTUALLY NOT TRUE. IT NEEDS TO INCORPORATE ALL FUTURES. SEE THE EXAMPLE OF THE $y^2 + 1$ TREE WHERE THE ROOT DIRECTIONS POINT TOWARDS THE CONSTANT TREE AND ITSELF; YOU CAN NEGLECT THE ENTIRE LEFT BRANCH IN A DENSE SUBSYSTEM.}
  Suppose there is such a state $s$ and sequence $I$;
  then, the subobject $U_{s,I} \subset G$ is nonzero and does not intersect $F$, so it is not dense.

  Now suppose that no such sequence exists, and suppose $B \subset G$ is a nonzero subobject containing a state $s \in B(i)$.
  Then, picking some sequence $I = i_0 \rightarrow i_1 \rightarrow \cdots$, we have $B \cap F \supset U_{s,I} \cap F \neq 0$, so $F$ is dense.
\end{proof}

The following dense monics will be useful.
\begin{example}
  Let $F$ be a $p$-dynamical system.
  Suppose $s \in F(i)$ is a state such that there is no endomorphism $f:i \rightarrow i$ sending $F(f)(s) = s$.
  Then, we may define the \emph{strict futures of $s$} by
  \[
    \widetilde U_s(j) := \cbr{t \in U_s(j) \mid t \neq s}.
  \]
  It quickly follows from Proposition \ref{Dense proposition} that the monic $U_{f_n(s),I_{\geq n}} \rightarrowtail U_{s,I}$ is dense.
\end{example}

One characterization of the closure of $F\subset G$ is the largest closed subobject of $G$ containing $F$ as a dense subobject.\footnote{\color{magenta} Is this true?}
Using this and Proposition \ref{Dense proposition}, we have the following lemma.
\begin{lemma}
  Let $F \subset G$ be a $p$-dynamical subsystem.
  Then, we may compute the closure of $F$ as \textbf{CHARACTERIZATION GOES HERE.} 
  \[
    \neg \neg F
  \]
\end{lemma}

\subsection{Characterizing sheaves when $p$ is linear.}
For now I'll cover a special case.
\begin{proposition}\label{Linear polynomial proposition}
  Suppose $p = Ay + B$ is a linear polynomial.
  Then, a $p$-dynamical system $F$ is a $\neg\neq$-sheaf iff every morphism $f:i \rightarrow j$ in $\sF_p$ induces a bijective transition function $F(f)$.
\end{proposition}

We'll heavily use the following lemma.
\begin{lemma}\label{Nonfixed lemma}
  Let $f:i \rightarrow j$ be a morphism in $\sF_p$ for $p$ a linear polynomial.
  \begin{enumerate}[label={(\roman*)}]
    \item Suppose $F(f)$ fails to be injective for some $f:i \rightarrow j$.
    Then, there is some morphism $h:i \rightarrow k$ and two distinct elements distinct elements $s,t \in F(i)$ such that $F(g)(s) \neq s$ for all non-identity endomorphisms $g:i \rightarrow i$ and $F(f)(s) = F(f)(t)$.
    \item Suppose $F(f)$ fails to be surjective for some $f:j \rightarrow i$.
      Then, there is some $h:k \rightarrow i$ and element $s \in F(i) - \operatorname{im} h$ such that $F(g)(s) \neq s$ for all non-identity endomorphisms $g:i \rightarrow i$.
  \end{enumerate}
\end{lemma}
\begin{proof}
  The lemma is immediate if there are no non-identity endomorphisms of $i$, so suppose that $i$ has such an endomorphism $g$.
  Since $p$ is linear, we have $\End(i) = \cbr{g^n}_{n \in \NN}$.
  
  Suppose contrapositively that all $s \in F(i)$ have some $n_s$ where $F(g)^{n_s}(s) = s$.
  Then, $s \in \operatorname{im} F(g)^{n_s} \subset \operatorname{im} F(g)$, so $F(g)$ is surjective.
  Conversely, if $s,t$ have $F(g)(s) = F(g)(t)$, then we have 
  \[
    s = F(g)^{n_sn_t}(s) = F(g)^{n_sn_t}(t) = t
  \]
  so $F(g)$ is bijective, as desired.
  This implies that $F(g^n)$ is bijective for each $n$.

  For condition (i), note that there is some composition $i \xrightarrow{f} j \rightarrow i$ expressing $f$ as a prefix of a bijection;
  hence $f$ is injective.

  Now, consider some $F(f):F(i) \rightarrow F(j)$ failing to be injective, so that $F(g^n)$ fails to be injective for each $n$, and let $S := \cbr{s \in F(i) \mid g^n(s) \neq s \; \forall n \geq 1}$.
  Note that $F(g)(F(i) - S) \subset F(i) - S$ and $F(g)|_{F(i) - S}$ is injective (by the above argument)
  Hence there must be some $s \in S$ and $t \in F(i)$ satisfying $F(g)(s) = F(g)(t)$, as desired.

  Let $F(f):F(j) \rightarrow F(i)$ fail to be surjective, so that $F(g^n)$ fails to be surjective for each $n$;
  a composition $i \rightarrow j \xrightarrow{f} i$ proves that the subset $S := \cbr{s \in F(i) \mid g^n(s) \neq s \; \forall n \geq 1}$ is nonempty.
  Recall that $F(g)(F(i) - S) \subset F(i) - S$, and note that $F(g)|_{F(i) - S}$ surjects onto $F(i) - S$;
  hence there must be some $s \in F(i) - S$ which is not in the image of $g$, as desired.
\end{proof}




\begin{proof}[Proof of Proposition \ref{Linear polynomial proposition}]
  First suppose that $F$ has bijective transition functions, and suppose that $A \rightarrowtail E$ is a dense monic and $\tilde \varphi:A \rightarrow F$ is a morphism of $p$-dynamical systems.
  Then, for each $s \in E(i)$, there is a morphism $f:i \rightarrow j$ such that $f(s) \in A(j)$;
  we define an extension $\varphi:E \rightarrow F$ by $\varphi(s) := F(f)^{-1}(\tilde \varphi(f(s)))$.
  This is the unique choice of $\varphi(s)$ such that $F(f)(\varphi(s)) = \varphi(E(f)(s))$ for any $f$, so it is well defined as the unique extension of $\tilde \varphi$ to $E$.

  Conversely, suppose first that $F(f)$ fails to be injective for some $f:i \rightarrow j$.
  Then, choose $s,t,h$ as in Lemma \ref{Nonfixed lemma}, and note that $U_s = U_t$ as $p$-dynamical systems;
  however, they are distinct subsystems of $F$.
  Further, note that there are factorizations (where the leftmost monic is dense)
  \[
    \begin{tikzcd}
      & U_s = U_t \arrow[dr,tail,"s"]\\
      \tilde U_s \arrow[ru, tail] \arrow[rd, tail] \arrow[rr,tail]
      & & F\\
      & U_s = U_t \arrow[ur, tail, "t"]
    \end{tikzcd}
  \]
  which prove that the map $\Hom(E,F) \rightarrow \Hom(A,F)$ is not injective.

  Next, suppose that $F(f)$ fails to be surjective for some $f:j \rightarrow i$ and choose some $s \in i$ as in Lemma \ref{Nonfixed lemma}.
  The system $U_s$ occurs as a dense subobject of some system $U$ who includes a state in position $h$;
  however, the morphism $U_s \rightarrowtail F$ doesn't extend to $U$, as desired.
\end{proof}




{\color{magenta}
  This is left here for archival purposes;
  this argument works for linear systems with states in trees that are ``non-referencial.''
\subsection{Sheaves as ancient systems which remember history}
We say that a $p$-dynamical system $F$ is \emph{ancient} if all transition functions $F(i) \rightarrow F(j)$ are epic, and say that $F$ \emph{remembers history} if $F(i) \rightarrow F(j)$ are monic.
\begin{proposition}
  A $p$-dynamical system $F$ is a $\neg\neg$-separated presheaf iff it remembers history.
  A $p$-dynamical system $F$ is a $\neg\neg$-sheaf iff it is ancient and remembers history.
\end{proposition}
\begin{proof}
  We first verify the statement for separated presheaves.
  We have to verify that, $F$ remembers history iff for all dense monics $A \rightarrowtail E$, the induced map
  \[
    \Hom_{\sC_p}(E,F) \rightarrow \Hom_{\sC_p}(A,F)
  \]
  is monic.

  First suppose that $F$ does not remember history, and there are two distinct states $s,t \in F(i)$ and a morphism $f:i \rightarrow j$ such that $f(s) = f(t)$.
  Then, since $U_s = U_t$ as (dense) subobjects of $U_{f(s)} = U_{f(t)}$, we have two distinct extensions
  \[
    \begin{tikzcd}
      U_{s} \arrow[rd,tail] \arrow[rrd,tail, bend left]\\
      & U_{f(s) = f(t)} \arrow[r,tail]& F\\
      U_{t} \arrow[ru,tail] \arrow[rru, bend right]
    \end{tikzcd}
  \]
  so that $F$ is not separated.

  Conversely, suppose that $F$ does remember history, and suppose we have two morphisms $\varphi,\varphi':E \rightarrow F$ which each restrict to $A$ identically.
  At position $j$ possessing morphism $f:j \rightarrow i$, this is represented by commuting parallel morphisms 
  \[
    \begin{tikzcd}
      E(i) \arrow[r]
      & F(i)\\
      E(j) \arrow[u] \arrow[r, transform canvas={yshift = .4ex}]  \arrow[r, transform canvas={yshift = -.4ex}]
      & F(j) \arrow[u,tail]
    \end{tikzcd}
  \]
  which must coincide since the morphism $F(j) \rightarrowtail F(i)$ is monic.\footnote{\color{magenta} This needs to be corrected later;
it isn't necessarily the case that any $E(i)$ is totally subsumed by $A$.
Instead, we have to make some argument using colimits, which I don't particularly want to do..}

  For $\neg\neg$-sheaves, it suffices to prove that the morphism $\Hom_{\sC_p}(E,F) \rightarrow \Hom_{\sC_p}(A,F)$ is epic iff $F$ is ancient.
  This will work by a similar argument;
  if $F$ is not ancient, we may pick a state $s \in \coprod_i F(i)$ and morphism $f:j \rightarrow i$ such that $F(j)$ is nonempty (say, containing $t \in F(j)$) and such that $s \not\in \operatorname{im} f$, and consider the embedding $U_s \rightarrowtail F$.
  Then, $U_s$ is isomorphic as a $p$-dynamical system to $U_{f(t)} \subset U_t$, but there is no morphism $U_t \rightarrow F$ extending the given morphism.

  Conversely, if $F$ is ancient, we probably have to assume AoC, and then just choose preimages to extend a morphism $A \rightarrow F$ to $E$.
  I'll fill this in fully later.

   
\end{proof}
}
\end{document}
