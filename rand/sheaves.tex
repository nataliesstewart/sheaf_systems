\documentclass{amsart}  
\usepackage{../catstyle} 
\usepackage{../../../environments}
\begin{document}

\title{On double-negation sheaves in the copresheaf topos on a cofree comonoid}
\author{Natalie Stewart}
\maketitle

This note fixes a polynomial functor $p \in \Poly$ and studies the copresheaf topos $\sC_p := \sF_P$-$\Set$;
we fix $\Omega \in \sC_p$ the subobject classifier (given by the copresheaf of sieves on $\sF_p^{\op}$).

\section{Presheaves}
\subsection{$\sC_p$ as a topos of dynamical systems}
A presheaf $F \in \sF_p$ consists of the following data (subject to no restrictions):
\begin{enumerate}
  \item For each $p$-tree $i \in \sF_p$, a set $F(i)$.
  \item For each morphism $f:i \rightarrow j$ in $\sF_p$ corresponding with an inclusion of a subtree at height one, a function $F(f):F(i) \rightarrow F(j)$. 
\end{enumerate}
This corresponds with a ``dynamical system'' with ``positions with future knowledge'' corresponding with $p$-trees, states in each position corresponding with each set $F(i)$, and ``transition functions'' between the states in these positions where one moves to a possible ``next state'' according to the futures predicted by the $p$-tree structure.\footnote{\color{magenta} This sentence makes no sense.}
I will henceforth refer to a presheaf $F \in \sF_p$ as a \emph{$p$-dynamical system}.

\subsection{Subsystems}
Note that a morphism of $p$-dynamical systems $F \rightarrow G$ is precisely a map $F(i) \rightarrow G(i)$ from the $i$-states of $F$ to the $i$-states of $G$ for each $i \in \sF_p$, compatible with the transition functions in $F$ and $G$.
A morphism is monic iff each of the constituent morphisms are monic;
that is, a subobject corresponds with a subset of the states which is preserved under the transition functions.

As with any topos, the set of subobjects on a $p$-dynamical system $\Sub(F)$ comes equipped with a Heyting algebra stricture;
the join and meet are given by union and intersection, and the implication $A \implies B$ is given by \textbf{FILL IN THE RELEVANT CHARACTERIZATION}.
In particular, the \emph{negation} $\neg A = (A \implies 0)$ is given by the largest dynamical system contained in the set-theoretic complement of $A$, and hence $\neg \neq A$ is given by the largest dynamical system whose complement is the same as $A$.

As with any topos, $\neg \neg:\Omega \rightarrow \Omega$ is a Lawvere-Tierney topology on $\sC_p$.
We will seek to characterize the sheaves with respect to this topology, for which it will be useful to name the following dynamical (sub)systems.

\begin{example}
  Let $s \in F(i)$ be a state in a dynamical system, and let $i = i_0 \rightarrow i_1 \rightarrow \cdots$ be an $\NN$-indexed sequence of composable morphisms.
  Define $I := (i_n)_{n \in \NN}$.\footnote{\color{magenta} The notation here is confusing, and should be fixed.}
  Then, we may define the dynamical system $U_{s,I}$, called the \emph{future of $s$ along $I$} by
  \[
    U_{s,I}(j) = \begin{cases}
      1 & \;\;\; j \in I,\\
      0 & \;\;\; \text{otherwise.} 
    \end{cases}
  \]
  The transition functions are canonically defined, and there is a unique monic $U_{s,I} \rightarrowtail F$ sending $U_{s,I}(i)$ to $S$.
  We may combine these to yield a subsystem of \emph{futures of $S$}: 
  \[
    U_s(j) := \bigcup_{I \in \prn{\sF_p}^{\NN} \text{ s.t. } I_0 = i} U_{s,I} = \begin{cases}
      1 & \;\;\; \exists \, \text{ morphism } i \rightarrow j,\\
      0 & \;\;\; \text{otherwise.} 
    \end{cases}
  \]

  \begin{lemma}
    Any dynamical system $F$ is generated by the associated subsystems $U_{s}$;
    that is, 
    \[
      F = \bigcup_{s \in \coprod_i F(i)} U_{s}.
    \]
  \end{lemma}

  Similarly, for a $\ZZ$-indexed sequence $I$ of composable morphisms with $I_0 = i$ and such that each morphism having codomain $i$ has image containing $s$, there is a system $V_{s,I}$ called the \emph{eternity of $s$ along $I$}, defined by
  \[
    V_s(j) = \begin{cases}
      1 & \;\;\; j \in I\\
      0 & \;\;\; \text{ otherwise.}
    \end{cases}
  \]
  and we may combine these into the \emph{eternities of $s$}: 
  \[
    V_s(j) := \bigcup_{I} V_{s,I} = \begin{cases}
      1 & \;\;\; \exists \, \text{ morphism } i \rightarrow j \text{ or } \; \exists \, \text{ morphism } f:j \rightarrow i \;\; \text{ s.t. } \,\, s \in \operatorname{im} f,\\
      0 & \;\;\; \text{ otherwise.}
    \end{cases}
  \]
  There 
\end{example}

It is perhaps troublesome that $U_s$ is defined in reference to a particular state of a dynamical system, as the systems themselves depend only on the position that the state resides in.
We can simply define $U_i$ for position $i$ by the above description;
these are contained in the following example.
\begin{example}
  The \emph{all-ones system} $1_p$ has $1_p(i) = \cbr{i}$ for all $i \in \sF_p$, with transition functions given by the unique endomorphism of 1.
  Note that there are unique monics $U_i \rightarrowtail V_i \rightarrowtail 1_p$.
  
  {\color{magenta} Consider mentioning that they generate all subobjects under union--they always do, but this time it's particularly simple.}
\end{example}


\section{Sheaves}\label{Sheaves section}
\subsection{Density}
The following lemma is well known:
\begin{lemma}
  In a presheaf topos, a subobject $A \subset C$ is $\neg\neg$-dense iff all nonzero subobjects $0 \subsetneq B \subset C$ intersect $A$.
\end{lemma}
This powers the following (easy) proposition, which shows that dense subsystems correspond intuitively with \emph{attainable win conditions}:
\begin{proposition}\label{Dense proposition}
  A subsystem $F \subset G$ is dense iff there is no state $s \in F(i)$ and sequence of morphisms $i = i_0 \xrightarrow{f_0} i_1 \xrightarrow{f_1} i_2 \xrightarrow{f_2} \cdots$ such that $f_n(s) \not \in G$ for all $n \in \NN$. 
\end{proposition}
\begin{proof}
  Suppose there is such a state $s$ and sequence $I$;
  then, the subobject $U_{s,I} \subset G$ is nonzero and does not intersect $F$, so it is not dense.

  Now suppose that no such sequence exists, and suppose $B \subset G$ is a nonzero subobject containing a state $s \in B(i)$.
  Then, picking some sequence $I = i_0 \rightarrow i_1 \rightarrow \cdots$, we have $B \cap F \supset U_{s,I} \cap F \neq 0$, so $F$ is dense.
\end{proof}

The following dense monics will be useful.
\begin{example}
  Let $F$ be a $p$-dynamical system.
  Let $I$ be the sequence $i_0 \xrightarrow{f_1} i_1 \xrightarrow{f_2} \cdots$, and let $I_{\geq n}$ be the suffix $i_n \rightarrow i_{n+1} \rightarrow \cdots$.
  Then, there is a factorization
  \[
      U_{f_n(s),I_{\geq n}} \rightarrowtail U_{s,I} \rightarrowtail F.
  \]
  It quickly follows from Proposition \ref{Dense proposition} that the monic $U_{f_n(s),I_{\geq n}} \rightarrowtail U_{s,I}$ is dense.

  Similarly, for $I$ a $\ZZ$-diagram and $I_{\geq 0}$ its truncation to $\NN$, the monic $U_{s,I_{\geq 0}} \rightarrowtail V_{s,I}$ is monic.
\end{example}

Using this and Proposition \ref{Dense proposition}, we have the following lemma.
\begin{lemma}
  Let $F \subset G$ be a $p$-dynamical subsystem.
  Then, the closure 
  \[
    a
  \]
\end{lemma}

\subsection{Sheaves as ancient systems which remember history}
We say that a $p$-dynamical system $F$ is \emph{ancient} if all transition functions $F(i) \rightarrow F(j)$ are epic, and say that $F$ \emph{remembers history} if $F(i) \rightarrow F(j)$ are monic.
\begin{proposition}
  A $p$-dynamical system $F$ is a $\neg\neg$-separated presheaf iff it remembers history.
  A $p$-dynamical system $F$ is a $\neg\neg$-sheaf iff it is ancient and remembers history.
\end{proposition}
\begin{proof}
\end{proof}

\end{document}
